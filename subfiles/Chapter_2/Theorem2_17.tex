\documentclass[../../main.tex]{subfiles}

\begin{document}
\subsection{$\sup A\leq \inf B\iff a\leq b$ for every $a\in A$, $b\in B$}
\begin{wts}\label{lemma: a leq b for every a and b}
    If $A$, $B$ are non-empty subsets of $\real$,
    \[
    \sup A\leq \inf B \iff \forall a\in A,\,\forall b\in B,\, a\leq b
    \]
\end{wts}
\begin{proof}
    Suppose that $\sup A\leq \inf B$, then for every $a\in A$, and we can safely assume that both $\sup A$ and $\inf B$ are finite (see remark),
    \[
    a\leq \sup A\leq \inf B
    \]
    so that $a$ is a lower bound for $B$, but this is equivalent to saying that $a\leq b$ for every $b\in B$.\\
    
    Now suppose that for every $a,b\in A,B$, $a\leq b$. Then every single $b\in B$ is an upper bound for the set $A$, therefore
    \[
    \forall b\in B,\, \sup A\leq b\implies \sup A\leq \inf B
    \]
    where the last estimate is due to $\sup A$ being a lower bound for $B$.
\end{proof}
\begin{remark}
    If $\sup A = +\infty$, then $\inf B = +\infty$, this can only happen if $B=\varnothing$, so $\sup A = +\infty$ is impossible, so is $\inf B = -\infty$. (We assume that $A$ and $B$ are subsets of $\real$, and not of $\cl{\real}$).\\
    
    Further, if $\sup A = -\infty$, then either $A=\{-\infty\}$ which is not a subset of $\real$, or $A=\varnothing$, which is again impossible.
\end{remark}

\end{document}