\documentclass[../../main.tex]{subfiles}

\begin{document}
\subsection{Theorem 4.4}
\begin{wts}
    If $\Epsilon\subseteq\mathbb{P}(X)$, the topology $\Tau(\Epsilon)$ generated by $\Epsilon$ consists of $\varnothing, X$ and all unions of finite intersections of $\Epsilon$, in symbols
    \[
    \Tau(\Epsilon)=\{\varnothing, X\}\cup\left\{\bigcup W_\alpha,\:W_\alpha = \bigcap E_{j\leq n},\: E_j\in\Epsilon\right\}
    \]
\end{wts}
\begin{proof}
    Denote the set
    \[
    W = \{X\}\cup\left\{\bigcap V_{j\leq n},\:\: V_j\in\Epsilon\right\}
    \]
    We claim this set $W$ satisfies Theorem 4.3. Since 4.3a) is satisfied with $X\in W$. 4.3b) follows since the right member in $W$ is closed under intersections.\\
    
    And if we are taking an element from each member, $E_1\in\{\varnothing,X\}$ and $E_2$ is an element in the right member, then it is trivial to verify that their intersection is always contained within $W$. Therefore $W$ induces a topology by Theorem 4.2, and we call this topology $\Tau$ — and for the sake of completeness
    \[
    \Tau = \left\{U\subseteq X,\: \forall x\in U,\:\exists V\in \Epsilon, \: x\in V\subseteq U\right\}
    \]
    We so claim that if we define $\cl{W}$ as the union of all members $w\in W$, together with the empty set, is equal to the set $\Tau$.
    \[
    \cl{W} = \left\{\bigcup_{w\in W}w\right\}\cup\{\varnothing\}
    \]
    \begin{itemize}
        \item We want to show $\Tau\subseteq\cl{W}$, since $W$ is a base for the topology $\Tau$, every (non-empty) $U\in \Tau$ is the union of members in $W$ (Theorem 4.2), and there exists some $B\subseteq W$ with
        \[
        U=\bigcup E_{\alpha\in B}\in \cl{W}
        \]
        Now if $U$ is the empty set then it is trivially contained within $\cl{W}$.
        \item Next, we show that $\cl{W}\subseteq\Tau$, fix any element $E\in\cl{W}$, if $E=\varnothing$ then there is nothing to prove since $\Tau$ is a topology. Now for every $x\in E$,
        \[
        x\in E=\bigcup_{w\in W} w\implies x\in w
        \]
        Therefore $E\in \Tau$ by definition. This proves that $\Tau = \cl{W}$.
    \end{itemize}
    Now that $\cl{W}$ is a topology, that contains $\Epsilon$ as a subset, and by definition of $\Tau(\Epsilon)$
    \[
    \Tau(\Epsilon) = \bigcap \left\{A,\text{ is a topology, and } \Epsilon\subseteq A \right\}
    \]
    Tells us
    \[
    \Tau(\Epsilon) \subseteq \cl{W}, \quad\text{ since }\cl{W}\in\left\{A,\text{ is a topology, and } \Epsilon\subseteq A \right\}
    \]
    Conversely, fix any member $E\in\cl{W}$, if $E=\varnothing$ then $E\in\Tau(\Epsilon)$, if not, then there exists some subset $B\subseteq W$ such that
    \[
    E = \bigcup_{w\in B}w=\bigcup_{w\in B} \bigcap_{j\leq n}V_{j\leq n}^w \, V_j\in \Epsilon\cup\{X\}
    \]
    Since $\Tau(\Epsilon)$ is closed under finite intersections and unions, and it contains $\Epsilon$ as a subset, $\cl{W} = \Tau(\Epsilon)$ and we are done.
\end{proof}

\end{document}