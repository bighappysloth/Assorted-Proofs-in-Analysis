\documentclass[../../main.tex]{subfiles}

\begin{document}
\subsection{Lemma}
\begin{lemma}\label{lemma:add_sup}
    If $A$ and $B$ are non-empty bounded above subsets of $\mathbb{R}$, then $\sup A +\sup B = \sup(A+B)$
\end{lemma}
\begin{proof}
    Define $s = \sup A$ and $t = \sup B$, then for every $(a,b)\in A\times B$
    \[
        a\leq s,\:b\leq t\implies a+b\leq s+t\implies A+B\leq s+t
    \]
    Now for every $\varepsilon/2>0$, there exists $(a,b)\in A\times B$ such that
    \[
    a\leq s-\varepsilon/2,\:b\leq t-\varepsilon/2\implies s+t-\varepsilon\leq a+b
    \]
    Therefore $\sup(A+B) = \sup(A) + \sup(B)$.
\end{proof}

\end{document}