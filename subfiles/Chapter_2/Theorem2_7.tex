\documentclass[../../main.tex]{subfiles}

\begin{document}
\subsection{Lemma}
\begin{lemma}\label{lemma:contained_sup_inf}
    If $A$ is a non-empty bounded subset of $\mathbb{R}$, if $s$ and $t$ are upper and lower bounds of $A$, and if $s\in A$ then $s=\sup A$. Also if $t\in A$, then $t = \inf A$
\end{lemma}
\begin{proof}
    Suppose that $s$ and $s'$ are upper bounds of $A$, then
    \[
    s\in A\implies s\leq s'
    \]
    So $s=\sup A$, now if $t$ and $t'$ are lower bounds of $A$, then
    \[
    t\in A\implies t'\leq t
    \]
    This completes the proof.
\end{proof}
\remark We only require $A$ to be bounded above for the supremum part of the proof, and $A$ to be bounded below for the infimum of the proof. We omit the caveat because it is a needless distraction.
\end{document}