\documentclass[../../main.tex]{subfiles}

\begin{document}
\subsection{Lemma}
\begin{lemma}\label{lemma:mult_with_sup}
    If $A$ is a non-empty bounded above subset of $\mathbb{R}$, then for every  $c\geq 0, c(\sup A) = \sup cA$
\end{lemma}
\begin{proof}
    Let $s = \sup(A)$, then 
    \[
    cA = \{cx: x\in A\}
    \]
    Then for every $x\in A$
    \[
    x\leq s\implies cx \leq cs\implies cA\leq cs
    \]
    If $c=0$, then the equality is trivial since $cA = \{0\}$, if not, for every $\varepsilon/c>0$, there exists an $x\in A$ such that
    \[
    s-\dfrac{\varepsilon}{c}<x\implies cx - \varepsilon< cx
    \]
    This establishes the Lemma.
\end{proof}

\end{document}